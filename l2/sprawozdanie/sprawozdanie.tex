% ***************************************************************************************************
%
%	Szablon pracy magisterskiej dla Politechniki Wrocławskiej w wersji dwustronnej.
%	Autor:	Tomasz Strzałka
%
% ***************************************************************************************************

% Styl dwustronny z domyślną wielkością czcionki 10pt oraz oddzieloną stroną tytułową (titlepage).
% Domyślnie rodziały rozpoczynają się na stronie prawej (openright).
\documentclass{book}

% ***************************************************************************************************
% Ustawienia języka
% ***************************************************************************************************

% Podstawowe ustawienia języka, według którego formatowany będzie dokument
\usepackage[polish]{babel}

% Pakiet babel dla polskiego języka powoduje konflikt z pakietem amssymb.
% Polecenie '\lll' definiują oba pakiety - porządana jest druga definicja.
\let\lll\undefined

% W przypadku wielojęzykowości ustawia główny język dokumentu
\selectlanguage{polish}

% Kodowanie dokumentu
\usepackage[utf8]{inputenc}

% Dowolny rozmiar czcionek, kodowanie znaków
\usepackage{lmodern}

% Polskie wcięcia akapitów
\usepackage{indentfirst}

% Polskie łamanie wyrazów
\usepackage[plmath]{polski}

% Przecinek w wyrażeniach matematycznych zamiast kropki
\usepackage{icomma}

% Polskie formatowanie typograficzne
\frenchspacing

% Zapewnia liczne usprawnienia wyświetlania i organizacji matematycznych formuł. 
\usepackage{amsmath}

% Wprowadza rozszerzony zestaw symboli m.in. \leadsto
\usepackage{amssymb}

% Dodatkowa, ,,kręcona'' czcionka matematyczna
\usepackage{mathrsfs}

% Dodatkowe wsparcie dla środowiska mathbb, które nie wspiera domyślnie cyfr (\mathbb{})
% \usepackage{bbold} TODO

% Fixes/improves amsmath
\usepackage{mathtools}


% ***************************************************************************************************
% Kolory  
% ***************************************************************************************************

% Umożliwia kolorowanie poszczególnych komórek tabeli
\usepackage[table]{xcolor}% http://ctan.org/pkg/

% Umożliwia łatwą zmianę koloru linii w tabeli
\usepackage{tabu}

% Umożliwia rozszerzoną kontrolę nad kolorami.
\usepackage{xcolor}

% Definicje kolorów
\definecolor{lgray}{HTML}{9F9F9F}
\definecolor{dgray}{HTML}{5F5F5F}
% lgray				-	nazwa nowo zdefiniowanego koloru
% HTML				-	model kolorów
% CCCCCC			-	wartość koloru zgodna z modelem

% ***************************************************************************************************
% Algorytmy 
% ***************************************************************************************************

% Udostępnia środowisko do konstruowania pseudokodów
\usepackage[ruled,vlined,linesnumbered,longend,algochapter]{algorithm2e}
% ruled	- poziome kreski na początku i końcu algorytmu, podpis na górze oddzielony również kreską poziomą
% vlined - pionowe kreski łączące początek polecenia z jego końcem
% linesnumbered	- numerowanie kolejnych wierszy algorytmu
% longend - długie końcówki np. ifend, forend itd.
% algochapter - numeracja z rozdziałami

% Zamiana nazwy środowiska z domyślnej "Algorithm X" na "Pseudokod X"
\newenvironment{pseudokod}[1][htb]{
	\renewcommand{\algorithmcfname}{Pseudokod}
	\begin{algorithm}[#1]%
	}{
\end{algorithm}
}

% Zmiana rozmiaru komentarzy
\newcommand\algcomment[1]{
	\footnotesize{#1}
}

% Ustawienie zadanego stylu dla komentarzy
\SetCommentSty{algcomment}

% Wyśrodkowana tylda
\usepackage{textcomp}%
\newcommand{\textapprox}{\raisebox{0.5ex}{\texttildelow}}

% Listowanie kodów źródłowych
\usepackage{listings} 
\renewcommand{\lstlistingname}{Kod źródłowy} % Polska nazwa listingu

% Definicje pecjalnych znaków, które nie są obsługiwane w środowisku listing
\lstset{literate=
	{ż}{{\.{z}}}1	{ź}{{\'{z}}}1
	{ć}{{\'{c}}}1	{ń}{{\'{n}}}1
	{ą}{{\c a}}1	{ś}{{\'{s}}}1
	{ł}{{\l}}1		{ę}{{\c{e}}}1
	{ó}{{\'{o}}}1	{á}{{\'a}}1
	{é}{{\'e}}1		{í}{{\'i}}1
	{ó}{{\'o}}1		{ú}{{\'u}}1
	{ù}{{\`u}}1		{Á}{{\'A}}1
	{É}{{\'E}}1		{Í}{{\'I}}1
	{Ó}{{\'O}}1		{Ú}{{\'U}}1
	{à}{{\`a}}1		{è}{{\'e}}1
	{ì}{{\`i}}1		{ò}{{\`o}}1
	{ò}{{\`o}}1		{À}{{\`A}}1
	{È}{{\'E}}1		{Ì}{{\`I}}1
	{Ò}{{\`O}}1		{Ò}{{\`O}}1
	{ä}{{\"a}}1		{ë}{{\"e}}1
	{ï}{{\"i}}1		{ö}{{\"o}}1
	{ü}{{\"u}}1		{Ä}{{\"A}}1
	{Ë}{{\"E}}1		{Ï}{{\"I}}1
	{Ö}{{\"O}}1		{Ü}{{\"U}}1
	{â}{{\^a}}1		{ê}{{\^e}}1
	{î}{{\^i}}1		{ô}{{\^o}}1
	{û}{{\^u}}1		{Â}{{\^A}}1
	{Ê}{{\^E}}1		{Î}{{\^I}}1
	{Ô}{{\^O}}1		{Û}{{\^U}}1
	{œ}{{\oe}}1		{Œ}{{\OE}}1
	{æ}{{\ae}}1		{Æ}{{\AE}}1
	{ß}{{\ss}}1		{ç}{{\c c}}1
	{Ç}{{\c C}}1	{ø}{{\o}}1
	{å}{{\r a}}1	{Å}{{\r A}}1
	{€}{{\EUR}}1	{£}{{\pounds}}1
}

% ***************************************************************************************************
% Marginesy 
% ***************************************************************************************************

% Ustawienia rozmiarów stron i ich marginesów
\usepackage[headheight=18pt, top=25mm, bottom=25mm, left=25mm, right=25mm]{geometry}
% headheight		-	wysokość tytułów
% top				-	margines górny
% bottom			-	margines dolny
% left				-	margines lewy
% right				-	margines prawy

% Usunięcie górnego marginesu dla środowisk
\makeatletter
\setlength\@fptop{0\p@}	
\makeatother

% ***************************************************************************************************
% Styl 
% ***************************************************************************************************

% Definiuje środowisko 'titlingpage', które zapewnia pełną kontrolę nad układem strony tytułowej.
\usepackage{titling}


% Umożliwia modyfikowanie stylu spisu treści
\usepackage{tocloft}	

\tocloftpagestyle{tableOfContentStyle}

% Definiowanie własnych stylów nagłówków i/lub stopek
\usepackage{fancyhdr}

% Umożliwia wstawianie hiperłączy do dokumentu
\usepackage{hyperref}							% Aktywuje linki

% Umożliwia zdefiniowanie własnego stylu wyliczeniowego
\usepackage{enumitem}

% Dołączanie rysunków
\usepackage{graphicx}

% Figury i przypisy
\usepackage{caption}
\usepackage{subcaption}

% Umożliwia tworzenie przypisów wewnątrz środowisk
\usepackage{footnote}

% Umożliwia tworzenie struktur katalogów
\usepackage{dirtree}

% Rozciąganie komórek tabeli na wiele wierszy
\usepackage{multirow}

% Precyzyjne obliczenia szerokości/wysokości dowolnego fragmentu wygenerowanego przez LaTeX
\usepackage{calc}

% Numerowanie z wieloma poziomami zagłębienia
\usepackage{outlines}

% Pozycjonowanie zdjęcia na stronie i opływanie go tekstem
\usepackage{float}
\usepackage{wrapfig}

% ***************************************************************************************************
% Matematyczne skróty
% ***************************************************************************************************

% Skrócony symbol liczb rzeczywistych
\newcommand{\RR}{\mathbb{R}}

% Skrócony symbol liczb naturalnych
\newcommand{\NN}{\mathbb{N}}

% Skrócony symbol liczb wymiernych
\newcommand{\QQ}{\mathbb{Q}}

% Skrócony symbol liczb całkowitych
\newcommand{\ZZ}{\mathbb{Z}}

% Skrócony symbol logicznej implikacji
\newcommand{\IMP}{\rightarrow}

% Skrócony symbol  logicznej równoważności
\newcommand{\IFF}{\leftrightarrow}

% ***************************************************************************************************
% Dokument
% ***************************************************************************************************

\frontmatter

\title{Sprawozdanie\\Wprowadzenie do Sztucznej Inteligencji\\Labolatoria 2}
\author{Kamil Matejuk}
\date{14.11.2021r}
\begin{document}
	\pagenumbering{arabic}
    \maketitle
    \section*{Opis heurystyk}
\begin{wrapfigure}{r}{0.15\textwidth}
    \centering
    \includegraphics[width=0.15\textwidth, height=0.15\textwidth]{manhattan-distance.png}
\end{wrapfigure}

Algorytmy oceny heurystycznej, które przetestowałem, opierały się na dystansie manhatańskim. W łamigłówce dystans ten jest liczony jako suma różnicy kolumn i różnicy wierszy.
\begin{align*}
    x\_curr & = \textit{kolumna w której znajduje się puzzel układanki} \\
    x\_target &= \textit{kolumna w której powinien znajdować się puzzel układanki} \\
    y\_curr &= \textit{wiersz w której znajduje się puzzel układanki} \\
    y\_target &= \textit{wiersz w której powinien znajdować się puzzel układanki} \\
    \\
    \textit{manhattan distance} &= abs(x\_curr - x\_target) + abs(y\_curr - y\_target)
\end{align*}
\\
\subsection*{Heurystyka 1}
Pierwsza haurystyka zliczała odległości każdego elementu do jego miejsca docelowego.
\[ manhattan(\textit{Puzzle p}) = abs(p.x - p.expected\_x) + abs(p.y - p.expected\_y) \]
\[ h1 = \sum_{p \in Puzzle} manhattan(p) \]

\subsection*{Heurystyka 2}
Druga heurystyka dodaje wagę do każdego dystansu na podstawie kolejności elementów, w taki sposób aby element $1$ w największym stopniu wypływał na wartość, natomiast element $15$ w najmniejszym. Celem takiego zabiegu było nakierowanie przeszukania tak, aby zacząć układać łamigłówkę od lewego-górnego rogu. 
\[ h2 = \sum_{p \in Puzzle} manhattan(p) * (16 - p.value) \]

\subsection*{Heurystyka 3}
Huerystyka 3 wykorzystuje wartość $h2$, natomiast dodaje odległość pustej kratki do pierwszego nie ustawionego na swoim miejscu elementu. Celem było preferowanie rozwiązań, które zbliżają się do ustawienia kolejnego elementu.
\[ h3 = \sum_{p \in Puzzle} \bigg( manhattan(p) * (16 - p.value) \bigg) + manhattan(\textit{first not ordered})\]


\section*{Metodyka testów}
Na początku testu generowana jest prawidłowo ułożona łamigłówka. Następnie jest ona mieszana wykonując $256$ poprawnych ruchów (rozmiar łamigłówki podniesiony do 4 potęgi). W ten sposób zapewnione jest, że łamigłówkę da się rozwiązać.\\
Następnie po kolei uruchamiany jest algorytm $A*$ z odpowiednią wersją algorytmu oceny heurystycznej. Każde uruchomienie rozpoczyna od tej samej łamigłówki początkowej, oraz ma ograniczony czas na rozwiązanie jej - 4 godziny. Z tego też względu nie każdy test zakończył się znalezieniem ścieżki, natomiast można go użyć do wyznaczenia dolnej granicy poszukiwanych wartości, ponieważ rzeczywiste wartości byłyby większe gdyby nie ograniczenia czasowe.
Każdy test został wykonany 10-krotnie.

\renewcommand{\arraystretch}{1.5}
\newcolumntype{M}[1]{>{\centering}m{#1}}

\section*{Wyniki}
\subsection*{Ukończone testy}
\phantom{.}\\
Średnie wartości z testów które ukończyły się we wskazanym czasie.\\

\begin{tabular}{ |M{3.5cm}|M{2cm}|M{2cm}|M{2cm}|c| } 
    \hline
    Wartości & h1 & h2 & h3 & \\
    \hline
    Ilość testów ukończonych w czasie & 8/10 & 3/10 & 9/10 & \\
    \hline
    & 77    & 164    & 186    & mediana \\
    Długość ścieżki & 77.75 & 184.66 & 190.88 & średnia \\
    & 9.24  & 75.72  & 29.26  & odchylenie standardowe \\
    \hline
    & 8549    & 3517    & 5155     & mediana \\
    Liczba odwiedzonych stanów & 8397.25 & 8725.33 & 10643.00 & średnia \\
    & 4449.07 & 8358.31 & 10801.06 & odchylenie standardowe \\
    \hline
    & 15m 23s &  2m 33s & 6m 35s     & mediana \\
    Czas wykonania & 18m 13s & 28m 49s & 43m 49s    & średnia \\
    & 15m 41s & 38m 12s & 1h 12m 44s & odchylenie standardowe \\
    \hline
\end{tabular}

\subsection*{Dolna granica}
\phantom{.}\\
Dolna granica wartości, wliczająca nieukończone testy. Dla testów które nie zakończyły się w czasie, jako długość ścieżki brana jest długość ścieżki (kandydata na rozwiązanie) ostatnio sprawdzanego stanu.\\

\begin{tabular}{ |M{3.5cm}|M{2cm}|M{2cm}|M{2cm}|c| } 
    \hline
    Wartości & h1 & h2 & h3 & \\
    \hline
    & 77    & 155    & 184    & mediana \\
    Długość ścieżki & 78.90 & 166.70 & 186.90 & średnia \\
    & 9.06  & 47.60  & 30.23  & odchylenie standardowe \\
    \hline
    & 9578     & 35364    & 7013     & mediana \\
    Liczba odwiedzonych stanów & 13395.40 & 27639.80 & 13193.20 & średnia \\
    & 10774.49 & 13211.72 & 12787.82 & odchylenie standardowe \\
    \hline
    & 18m 14s    & 4h         & 11m 33s    & mediana \\
    Czas wykonania & 1h 2m 34s  & 2h 56m 38s & 1h 3m 26s  & średnia \\
    & 1h 29m 48s & 1h 39m 1s  & 1h 30m 41s & odchylenie standardowe \\
    \hline
\end{tabular}

\vspace{3cm}

\subsection*{Porównanie}
\phantom{.}\\
Porównanie na mniejszych danych - dla łamigłówki 3x3. Pozwala to na puszczenie większej liczy testów - 100 iteracji na każdą heurystykę.\\

\begin{tabular}{ |M{3.5cm}|M{2cm}|M{2cm}|M{2cm}|c| } 
    \hline
    Wartości & h1 & h2 & h3 & \\
    \hline
                    & 28    & 49    & 52 & mediana \\
    Długość ścieżki & 28.60 & 50.00 & 53.20 & średnia \\
                    & 4.38  & 10.88 & 18.63 & odchylenie standardowe \\
    \hline
                               & 466    & 605    & 253    & mediana \\
    Liczba odwiedzonych stanów & 746.10 & 807.90 & 440.20 & średnia \\
                               & 704.09 & 524.26 & 412.44 & odchylenie standardowe \\
    \hline
                   & 1.2s  & 2.0s & 0.4s & mediana \\
    Czas wykonania & 5.6s  & 5.0s & 1.9s & średnia \\
                   & 10.3s & 5.4s & 2.4s & odchylenie standardowe \\
    \hline
\end{tabular}

\subsection*{Wnioski}
Da się zauważyć bardzo dużą różnorodność w wynikach (odchylenie standardowe jest bardzo duże). Mimo to, korzystając ze średniej oraz mediany można stwierdzić, iż algorytm 1 (najprostszy) pozwala na znalezienie najkrótszej ścieżki (około 77 ruchów przy 256 ruchach mieszających łamigłówkę). Algorytm 2 znajduje ścieżkę dłuższą, jednak nadal krótszą niż pomieszanie łamigłówki (około 190 ruchów przy 256 ruchach mieszających łamigłówkę).
Algorytm 2 jest znacznie dłuższy niż pozostałe 2, ponieważ większość uruchomień sprawdzało ponad 35000 stanów i trwało ponad 4h, mimo to nie znajdując rozwiązania.\\
Ponadto algorytm 1 jest znacznie bardziej stabilny niż algorytm 3. Odchylenie standardowe ilości odwiedzonych stanów algorytmu 1 wynosi 53\%, natomiast algorytmu 3 ponad 101\%.\\
Zatem najlepszym z porównywanych algorytmem oceny heurystycznej jest algorytm 1.

\end{document}

